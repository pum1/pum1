% Standardinkluderingsfil
%
%  untitled
%
%  Created by David Granqvist on 2008-09-08.
%  Modified by Martin Erola
%

% Set document format/class
\documentclass[a4paper,twoside]{article}

%%%%%%%%%%%%%%%%%%%
% Include packages
%
\usepackage[utf8]{inputenc}   % Use utf-8 encoding for foreign characters
\usepackage[swedish]{babel}   % Support for swedish letters
\usepackage{fullpage}         % Setup for fullpage use
\usepackage{fancyhdr}         % Running Headers and footers
\usepackage{boxedminipage}    % Surround parts of graphics with box
\usepackage{listings}         % Package for including code in the document
\usepackage{ifpdf}            % Recommended way for checking for PDFLaTeX:
\usepackage{tabularx}         % Tabeller med automatisk stretch
% \usepackage[nofancy]{svninfo} % Extract Subversion info about the file
% \usepackage{color}          % Color
% \usepackage{lastpage}       % Total page count

% Graphics
\ifpdf
\usepackage[pdftex]{graphicx}
\else
\usepackage{graphicx}
\fi
\usepackage{float}
%%%%%%%%%%%%%%%%%%%%%%%%%%%%%%%%%%%%%%%%%%%%%%%%%%%%%%%%%%
% Uncomment some of the following if you use the features
%

% Multipart figures
%\usepackage{subfigure}

% More symbols
%\usepackage{amsmath}
%\usepackage{amssymb}
%\usepackage{latexsym}

% If you want to generate a toc for each chapter (use with book)
% \usepackage{minitoc}

%%%%%%%%%%%%%%%%%%%%
% Document settings
%

% Header
\pagestyle{fancy}
% Sätter en marginal mellan header och (ovanstående?) text %
\setlength\headsep{10pt}
% Sätter höjden på headern
\setlength{\headheight}{32pt}

% Sätter styckesinställningar
\setlength\parindent{0pt}
\setlength\parskip{0pt}



\ifpdf
  \DeclareGraphicsExtensions{.pdf, .jpg, .tif, .png}
  \pdfinfo{            
    /Title  (Develop Technical Vision)
    /Author (PUM-grupp 1)
  }
\else
  \DeclareGraphicsExtensions{.eps, .jpg}
\fi

\title{Develop Technical Vision}
\author{PUM-grupp 1}
\date{\today}

\begin{document}

\maketitle\thispagestyle{empty}
\newpage

{\centering \Large{Dokumenthistorik\\}}

\vspace{10pt}
\begin{tabularx}{\textwidth}{ |l|l|X|l|l| }
  \hline
    \textbf{version} & \textbf{datum} & \textbf{utförda ändringar} & \textbf{utförda av} & \textbf{granskad} \\
	\hline 
  1.0 & 2009-02-12 &  Första versionen klar för inlämning  & Alla & Alla   \\
  \hline
\end{tabularx}

\newpage

\setcounter{tocdepth}{2}
\tableofcontents
\newpage

\section{Intressenter}
De intressenter vi har identiferat.
\begin{itemize}
\item Utvecklare, PUM-gruppen
\item Kunden, VISIARC AB
\item Användare
\begin{itemize}
\item Användare som ej har tillgång till en traditionell klient-server lösning
\item Användare som väljer en distribuerad lösning
\end{itemize}
\item Andra utvecklare
\end{itemize}
En beskrivning av varje intressent för att klargöra ansvar  
\begin{itemize}
\item PUM-gruppen är de som kommer utveckla systemet, gör projektnära beslut och implementera funktionerna kunden formulerar.  PUM-gruppen bör ha nära kontakt med kunden.
\item Kunden, VISIARC AB är den som formulerar kraven för systemet. 
\item Användare som ej har tillgång till en traditonell klient-server lösning. Vi ser att det finns två sorters användare i den här gruppen. De som av ekonomiska skäl väljer bort en serverlösning. Serverar är konstsamma i el, underhåll och lokal. Men också användare utan större datorvana de det för en oerfaren datoranvändare kan det vara svårt att få tillgång till en server. 
\item Användare som väljer en distribuerad lösning på grund av strukuren och rubustheten på systemet. Ett distribuerat system är att föredra om användarna arbetar mycket offline eller i otillförlitliga nätverk. Robustheten blir intressant då användaren ska publicera material som är kontroversiellt eller populärt då informationen blir redundant och har hög tillgänglighet.
\item Bland utvecklare så måste hänsyn tas både till de som kommer in i projeket i framtiden men också utvecklare inom helt andra projekt.
\end{itemize}
\section{Intressenters krav}
PUM-gruppen och Kunden har träffats och diskuterat. Den absolut enskilt viktigaste egenskapen för projektet är att programmet ska vara lätt att använda och lätt att installera men fortfarande vara användbart.  Vi riktar oss i början mot grupper om 5-10 personer utan större datorvana som delar samma wiki. En möjlig uppdelening på projektet är wiki, distribuerat versionshanteringsystem och peer-to-peer-nätverk. Några frågor som måste fattas beslut i är bland annat vilken peer-to-peer implementation vi ska använda, hur rättigheter ska implementeras i wiki:n, hur wiki:n ska editeras och läsas, om detta sker i en egen applikation eller genom webbläsare, vilken eller vilka licenser vi ska släppa programmet i samt vilket programmeringspråk vi ska implemtera programmet i.
\end{document}